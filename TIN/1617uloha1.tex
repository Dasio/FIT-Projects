% Šáblona prevzatá od Mareka Milkoviča
% https://github.com/metthal/TIN-Ulohy

\documentclass[12pt]{article}

\usepackage[margin=1in]{geometry} 
\usepackage{amsmath,amsthm,amssymb,amsfonts}
\usepackage[slovak]{babel}
\usepackage[utf8]{inputenc}
\usepackage{enumitem}
\usepackage{ifthen}
\usepackage{tikz}
\usepackage{tikz-qtree}
\usepackage{caption}
\usepackage{background}
\usepackage{float}

\usetikzlibrary{arrows,automata,calc,positioning}

\newcommand{\task}[2]{\par \noindent \textbf{{#1}.} \hspace{3pt} #2 \vspace{10pt}}
\newcommand{\solution}{\vspace{10pt}}
\newcommand{\pipesep}{\hspace{3pt} \vert \hspace{3pt}}

\newenvironment{subtasklist}[0]{\begin{enumerate}[label=(\alph*)]}{\end{enumerate}}
\newenvironment{mysolution}[1]{
    \par \textbf{Riešenie} \newline
    \ifthenelse{\equal{#1}{subtasks}}{\begin{enumerate}[label=(\alph*)]}
            {\begin{enumerate}[label={}] \item}
}{\end{enumerate} \newpage}
\newcommand{\subtask}{\item}

\SetBgContents{
        \parbox{0.4\textwidth}{
            \raggedleft
                Dávid Mikuš\\
                \texttt{xmikus15}
        }
}
\SetBgScale{0.75}
\SetBgOpacity{1}
\SetBgAngle{0}
\SetBgColor{black}
\SetBgPosition{current page.north east}
\SetBgVshift{-0.8cm}
\SetBgHshift{-3.5cm}
 
\begin{document}
 
\title{TIN 2016/2017: Úloha 1}
\author{Dávid Mikuš \\ \small\texttt{xmikus15@stud.fit.vutbr.cz}}
\maketitle

\task{1}{Uvažujte jazyk $L_{1} = \{a^{i}b^{j}c^{k} \pipesep ( i \ne k ) \lor (j=2l, l \ge 0)\}$.
    \begin{subtasklist}
            \subtask Zostavte gramatiku $G_{1}$ takú, že $L(G_{1}) = L_{1}$.
            \subtask Akého typu (podľa Chomského hierachie jazykov) je $G_{1}$ a akého typu
            je $L_{1}$? Môžu sa tieto typy všeobecne líšiť? Svoje tvrdenie zdôvodnite
            (formálny dôkaz nie je požadovaný).
    \end{subtasklist}}

\begin{mysolution}{subtasks}
\vspace{-15pt}
\subtask % (a)
\begin{flalign*}
    G_{1} = (&N, \Sigma, P, S) &\\[5pt]
    N &= \{S, A, B, D, E, F, G\} &\\
    \Sigma &= \{a, b, c\} &\\
    P&: \\
    &S \to D C c \pipesep a E \pipesep A F C &\\
    &A \to a A \pipesep \varepsilon &\\
    &B \to b B \pipesep \varepsilon &\\
    &C \to c C \pipesep \varepsilon &\\
    &D \to a D c \pipesep B \pipesep \varepsilon &\\
    &E \to a E G \pipesep B \pipesep \varepsilon &\\
    &F \to b b F \pipesep \varepsilon &\\
    &G \to c \pipesep \varepsilon &
\end{flalign*}

\subtask % (b)
Gramatika $G_{1}$ je typu 2 pretože obsahuje pravidlá tvaru $A \to \alpha$ kde $A \in N$ a $\alpha \in (N \cup \Sigma)^{*}$. Typu 3 už nemôže byť pretože pravidlá porušuju tvar $A \to xB \pipesep x$ kde $A,B \in N$ a $x \in \Sigma^{*}$
\\
Jazyk $L_{1}$ je typu 2, pretože už nemôže byť typom 3 (regulárny jazyk). \\
Tieto typy sa môžu všeobecne líšiť. Gramatika môže mať rovnaký typ ako jazyk alebo byť vyššieho typu v porovnaní s jazykom ktorý generuje. Napríklad bezkontextová gramatika (Typ 2) môže generovať regulárny jazyk (Typ 3).

\end{mysolution}

\task{2}{Uvažujte regulárny výraz $r_{2} = (a + b + \varepsilon)^{*}(ab)^{*}$.
    \begin{subtasklist}
        \subtask Preveďte $r_{2}$ algoritmicky na redukovaný deterministický konečný automat
        $M_{2}$ (tj. RV~$\to$ RKA $\to$ DKA $\to$ redukovaný DKA), prijímajúci jazyk
        popísaný výrazom $r_{2}$.
        \subtask Pre jazyk $L(M_{2})$ určte počet tried ekvivalencie relácie $\sim_{L}$
        (viz. Myhill-Nerodova veta) a vypíšte tieto triedy. Jednotlivé triedy môžete popísať
        napríklad konečným automatom, ktorý akceptuje všetky slová patriace do danej triedy.
    \end{subtasklist}
}

\begin{mysolution}{subtasks}
\vspace{-15pt}
\subtask % (a)
Použijeme algoritmus 3.7 zo študijnej opory. Najskôr rozložíme $r_{2}$ na primitívne zložky a rozklad vyjadríme stromom.

 \begin{figure}[!h]
        \centering
        \begin{tikzpicture}[level 1/.style={sibling distance = 3.75cm}, level 2/.style={sibling distance = 1.5cm}]
            \node {$r_{2}$}
            	child
            	{
            		node {$r_{A}$} edge from parent
            			child
            			{
            				node {$r_{C}$} edge from parent
            					child { node {$($} edge from parent }
            					child
            					{
            						node {$r_{E}$} edge from parent
            							child
            							{
            								node {$r_{G}$} edge from parent
            									child
            									{
            										node {$r_{K}$} edge from parent
            											child { node {$a$} edge from parent }
            									}
            									child { node {$+$} edge from parent }
            									child
            									{
            										node {$r_{L}$} edge from parent
            											child { node {$b$} edge from parent }
            									}
            							}
            							child { node {$+$} edge from parent }
            							child
            							{
            								node {$r_{H}$} edge from parent
            									child { node {$\varepsilon$} edge from parent }
            							};
            					}
            					child { node {$)$} edge from parent };
            			}                      
            			child { node {$\ast$} edge from parent };
            	}
            	child { node {\LARGE$\cdot$} edge from parent }
            	child
            	{
            		node {$r_{B}$} edge from parent
            			child
            			{
            				node {$r_{D}$} edge from parent
            					child { node {$($} edge from parent }
            					child
            					{
            						node {$r_{F}$} edge from parent
            							child
            							{
            								node {$r_{I}$} edge from parent
            									child { node {$a$} edge from parent }
            							}			
            					child { node {\LARGE$\cdot$} edge from parent }
            					child {
            						node {$r_{J}$} edge from parent
            							child { node {$b$} edge from parent }
            				}
            		}
            		child { node {$)$} edge from parent };
            			}
            			child { node {$\ast$} edge from parent }
            	};
        \end{tikzpicture}
        \caption{Rozklad regulárneho výrazu $r_{2}$.}
    \end{figure}
    
Následne zostrojíme ku každému regulárnemu výrazu príslušny konečny automat.

	\begin{figure}[!h]
        \centering
        \begin{minipage}[b]{0.45\textwidth}
            \centering
            \captionsetup{justification=centering}
            \begin{tikzpicture}[node distance = 3cm, ->, >=stealth, line width=1pt]
                \node[state, initial, initial text=] (s) {};
                \node[state, accepting, right of = s] (f) {};

                \path (s) edge node[above]{$\varepsilon$} (f);
            \end{tikzpicture}
            \caption{RKA $M_{H}$ ekvivalentný\\RV $r_{H}$.}
        \end{minipage}
        \begin{minipage}[b]{0.45\textwidth}
            \centering
            \captionsetup{justification=centering}
            \begin{tikzpicture}[node distance = 3cm, ->, >=stealth, line width=1pt]
                \node[state, initial, initial text=] (s) {};
                \node[state, accepting, right of = s] (f) {};

                \path (s) edge node[above]{$a$} (f);
            \end{tikzpicture}
            \caption{RKA $M_{I}, M_{K}$ ekvivalentný\\RV $r_{I}, r_{K}$.}
        \end{minipage}
    \end{figure}
    
    \begin{figure}[!h]
        \centering
        \begin{tikzpicture}[node distance = 3cm, ->, >=stealth, line width=1pt]
          		 \node[state, initial, initial text=] (s) {};
                \node[state, accepting, right of = s] (f) {};

                \path (s) edge node[above]{$b$} (f);
        \end{tikzpicture}
        \caption{RKA $M_{J}, M_{L}$ ekvivalentný RV $r_{J}, r_{L}$.}
    \end{figure}
    
    \begin{figure}[!h]
        \centering
        \begin{tikzpicture}[node distance = 2.5cm, ->, >=stealth, line width=1pt, scale=0.8, every node/.style={transform shape}]
            \node[state, initial, initial text=] (sy) {};
            \node[state, above right = 0.5cm and 1.7cm of sy] (sa) {};
            \node[state, right of = sa] (fa) {};
            \node[state, below right = 0.5cm and 1.7cm of sy] (sb) {};
            \node[state, right of = sb] (fe) {};
            \node[state, accepting, right = 6.5cm of sy] (fy) {};

            \path (sy) edge node[above]{$\varepsilon$} (sa);
            \path (sy) edge node[below]{$\varepsilon$} (sb);
            \path (sa) edge node[above]{$a$} (fa);
            \path (sb) edge node[above]{$b$} (fe);
            \path (fa) edge node[above]{$\varepsilon$} (fy);
            \path (fe) edge node[below]{$\varepsilon$} (fy);
        \end{tikzpicture}
        \caption{RKA $M_{G}$ ekvivalentný RV $r_{G}$.}
    \end{figure}
    
    \begin{figure}[!h]
        \centering
        \begin{tikzpicture}[node distance = 2.5cm, ->, >=stealth, line width=1pt, scale=0.8, every node/.style={transform shape}]
        	\node[state, initial, initial text=] (se) {};
            \node[state, above right = 0.5cm and 1.7cm of se] (sg) {};
            \node[state, above right = 0.5cm and 1.7cm of sg] (si) {};
            \node[state, right of = si] (fi) {};
            \node[state, below right = 0.5cm and 1.7cm of sg] (sb) {};
            \node[state, right of = sb] (fe) {};
            \node[state, right = 6.5cm of sg] (fy) {};
            \node[state, below = 3cm of si] (h) {};
            \node[state, right of = h] (fh) {};
           	\node[state, accepting, below right = 0.5cm and 1.7cm of fy] (fg) {};

			\path (se) edge node[above]{$\varepsilon$} (sg);
            \path (sg) edge node[above]{$\varepsilon$} (si);
            \path (sg) edge node[below]{$\varepsilon$} (sb);
            \path (si) edge node[above]{$a$} (fi);
            \path (sb) edge node[above]{$b$} (fe);
            \path (fi) edge node[above]{$\varepsilon$} (fy);
            \path (fe) edge node[below]{$\varepsilon$} (fy);
            \path (se) edge node[above]{$\varepsilon$} (h);
            \path (h) edge node[above]{$\varepsilon$} (fh);
            \path (fy) edge node[above]{$\varepsilon$} (fg);
            \path (fh) edge node[above]{$\varepsilon$} (fg);
        \end{tikzpicture}
        \caption{RKA $M_{E}, M_{C}$ ekvivalentný RV $r_{E}, r_{C}$.}
    \end{figure}
    
    \begin{figure}[!h]
        \centering
        \begin{tikzpicture}[node distance = 2cm, ->, >=stealth, line width=1pt]
            \node[state, initial, initial text=] (sa) {};
            \node[state, right of = sa] (fa) {};
            \node[state, accepting, right of = fa] (fb) {};

            \path (sa) edge node[above]{$a$} (fa);
            \path (fa) edge node[above]{$b$} (fb);
        \end{tikzpicture}
        \caption{RKA $M_{F}, M_{D}$ ekvivalentný RV $r_{F}, r_{D}$.}
    \end{figure}
    
    \begin{figure}[!h]
        \centering
        \begin{tikzpicture}[node distance = 2cm, ->, >=stealth, line width=1pt]
        	\node[state, initial, initial text=] (mb) {};
            \node[state, right of = mb] (sa) {};
            \node[state, right of = sa] (fa) {};
            \node[state, right of = fa] (fb) {};
            \node[state, accepting, right of = fb] (nb) {};

			\path (mb) edge node[above]{$\varepsilon$} (sa);
            \path (sa) edge node[above]{$a$} (fa);
            \path (fa) edge node[above]{$b$} (fb);
            \path (fb) edge node[above]{$\varepsilon$} (nb);
            \path (fb) edge[bend right=40] node[above]{$\varepsilon$} (sa);
            \path (mb) edge[bend right=40] node[below]{$\varepsilon$} (nb);
            
        \end{tikzpicture}
        \caption{RKA $M_{B}$, ekvivalentný RV $r_{B}$.}
    \end{figure}
    
    
    \begin{figure}[!h]
        \centering
        \begin{tikzpicture}[node distance = 2.5cm, ->, >=stealth, line width=1pt, scale=0.8, every node/.style={transform shape}]
        	\node[state, initial, initial text=] (ma) {};
        	\node[state, right of = ma] (se) {};
            \node[state, above right = 0.5cm and 1.7cm of se] (sg) {};
            \node[state, above right = 0.5cm and 1.7cm of sg] (si) {};
            \node[state, right of = si] (fi) {};
            \node[state, below right = 0.5cm and 1.7cm of sg] (sb) {};
            \node[state, right of = sb] (fe) {};
            \node[state, right = 6.5cm of sg] (fy) {};
            \node[state, below = 3cm of si] (h) {};
            \node[state, right of = h] (fh) {};
           	\node[state, below right = 0.5cm and 1.7cm of fy] (fg) {};
           	\node[state, accepting, right of = fg] (ng) {};
			
			\path (ma) edge[bend left=80] node[above]{$\varepsilon$} (ng);
			\path (ma) edge node[above]{$\varepsilon$} (se);
			\path (se) edge node[above]{$\varepsilon$} (sg);
            \path (sg) edge node[above]{$\varepsilon$} (si);
            \path (sg) edge node[below]{$\varepsilon$} (sb);
            \path (si) edge node[above]{$a$} (fi);
            \path (sb) edge node[above]{$b$} (fe);
            \path (fi) edge node[above]{$\varepsilon$} (fy);
            \path (fe) edge node[below]{$\varepsilon$} (fy);
            \path (se) edge node[above]{$\varepsilon$} (h);
            \path (h) edge node[above]{$\varepsilon$} (fh);
            \path (fy) edge node[above]{$\varepsilon$} (fg);
            \path (fh) edge node[above]{$\varepsilon$} (fg);
            \path (fg) edge node[above]{$\varepsilon$} (ng);
            \path (fg) edge[bend right=70] node[below]{$\varepsilon$} (se);
            
        \end{tikzpicture}
        \caption{RKA $M_{A}$ ekvivalentný RV $r_{A}$.}
    \end{figure}
    
    \begin{figure}[!h]
        \centering
        \begin{tikzpicture}[node distance = 2.5cm, ->, >=stealth, line width=1pt, scale=0.8, every node/.style={transform shape}]
        	\node[state, initial, initial text=] (ma) {$1$};
        	\node[state, right of = ma] (se) {$2$};
            \node[state, above right = 0.5cm and 1.7cm of se] (sg) {$3$};
            \node[state, above right = 0.5cm and 1.7cm of sg] (si) {$4$};
            \node[state, right of = si] (fi) {$5$};
            \node[state, below right = 0.5cm and 1.7cm of sg] (sb) {$6$};
            \node[state, right of = sb] (fe) {$7$};
            \node[state, right = 6.5cm of sg] (fy) {$8$};
            \node[state, below = 3cm of si] (h) {$9$};
            \node[state, right of = h] (fh) {$10$};
           	\node[state, below right = 0.5cm and 1.7cm of fy] (fg) {$11$};
			
			 \node[state, below = 4cm of sa] (mb) {$12$};
            \node[state, right of = mb] (sa) {$13$};
            \node[state, right of = sa] (fa) {$14$};
            \node[state, right of = fa] (fb) {$15$};
            \node[state, accepting, right of = fb] (nb) {$16$};
            
			\path (ma) edge[bend right=5] node[below]{$\varepsilon$} (mb);
			\path (ma) edge node[above]{$\varepsilon$} (se);
			\path (se) edge node[above]{$\varepsilon$} (sg);
            \path (sg) edge node[above]{$\varepsilon$} (si);
            \path (sg) edge node[below]{$\varepsilon$} (sb);
            \path (si) edge node[above]{$a$} (fi);
            \path (sb) edge node[above]{$b$} (fe);
            \path (fi) edge node[above]{$\varepsilon$} (fy);
            \path (fe) edge node[below]{$\varepsilon$} (fy);
            \path (se) edge node[above]{$\varepsilon$} (h);
            \path (h) edge node[above]{$\varepsilon$} (fh);
            \path (fy) edge node[above]{$\varepsilon$} (fg);
            \path (fh) edge node[above]{$\varepsilon$} (fg);
            \path (fg) edge[bend right=70] node[below]{$\varepsilon$} (se);
            
           

			\path (mb) edge node[above]{$\varepsilon$} (sa);
            \path (sa) edge node[above]{$a$} (fa);
            \path (fa) edge node[above]{$b$} (fb);
            \path (fb) edge node[above]{$\varepsilon$} (nb);
            \path (fb) edge[bend right=40] node[above]{$\varepsilon$} (sa);
            \path (mb) edge[bend right=40] node[below]{$\varepsilon$} (nb);
            
            \path (fg) edge[bend left=5] node[below]{$\varepsilon$} (mb);
        \end{tikzpicture}
        \caption{RKA $M_{2}$ ekvivalentný RV $r_{2}$.}
    \end{figure}

\clearpage
Podľa algoritmu 3.6 prevedieme RKA $M_{2}$ na DKA $M_{D} = (Q_{D}, \Sigma, \delta_{D}, q^{0}_{D}, F_{D})$.

\begin{flalign*}
        &q^{0}_{D} = \varepsilon\text{-uzáver}(1)
            = \{1, 2, 3, 4, 6, 9, 10, 11, 12, 13, 16\} = A & \\
                                                             & \\
        &\delta_{D}(A, a) = \varepsilon\text{-uzáver}(\{5, 14\})
            = \{2, 3, 4, 5, 6, 8, 9, 10, 11, 12, 13, 14, 16\} = B & \\
        &\delta_{D}(A, b) = \varepsilon\text{-uzáver}(\{7\})
        	= \{2, 3, 4, 6, 7, 8, 9, 10, 11, 12, 13, 16\} = C & \\
        &\delta_{D}(B, a) = \varepsilon\text{-uzáver}(\{5, 14\})
            = B & \\
       &\delta_{D}(B, b) = \varepsilon\text{-uzáver}(\{7, 15\})
       		= \{2, 3, 4, 6, 7, 8, 9, 10, 11, 12, 13, 15, 16\} = D & \\
       &\delta_{D}(C, a) = \varepsilon\text{-uzáver}(\{5, 14\})
             = B & \\
       &\delta_{D}(C, b) = \varepsilon\text{-uzáver}(\{7\})
            = C & \\
       &\delta_{D}(D, a) = \varepsilon\text{-uzáver}(\{5, 14\})
            = B & \\
        &\delta_{D}(D, b) = \varepsilon\text{-uzáver}(\{7\}) = C & \\
        & \\
        &Q_{D} = \{A, B, C, D\} & \\
        &F_{D} = \{A, B, C, D\}
    \end{flalign*}
    

    \begin{figure}[!h]
        \centering
        \begin{tikzpicture}[node distance = 2.2cm, ->, >=stealth, line width=1pt]
            \node[state, accepting, initial, initial text=] (a) {$A$};
            \node[state, accepting, right of = a] (b) {$B$};
            \node[state, accepting, below of = a] (c) {$C$};
            \node[state, accepting, right of = c] (d) {$D$};
          
            \path (a) edge node[above]{$a$} (b);
            \path (a) edge node[left]{$b$} (c);
            \path (b) edge[loop right] node[right]{$a$} (b);
            \path (b) edge[bend right=20] node[left]{$b$} (d);
            \path (c) edge node[left]{$a$} (b);
            \path (c) edge[loop left] node[left]{$b$} (c);
            \path (d) edge[bend right=20] node[right]{$a$} (b);
            \path (d) edge node[above]{$b$} (c);
           
        \end{tikzpicture}
        \caption{DKA $M_{D}$.}
    \end{figure}

Pomocou algoritmu 3.5 vykonáme prevod z DKA na redukovaný DKA.

\begin{figure}[!h]
        \begin{minipage}[b]{0.45\textwidth}
            \begin{center}
                \begin{tabular}{c  c | c  c}
                    $\overset{0}{\equiv}$ & $\delta_{U}$ & $a$ & $b$ \\
                    \hline
                    $I:$   & $A$ & $B_{I}$ & $C_{I}$ \\
                           & $B$ & $B_{I}$ & $D_{I}$ \\
                           & $C$ & $B_{I}$ & $C_{I}$ \\
                           & $D$ & $B_{I}$ & $C_{I}$ \\
                    \hline
                \end{tabular}
            \end{center}
        \end{minipage}
        \begin{minipage}[b]{0.45\textwidth}
            \begin{flalign*}
                &\overset{0}{\equiv}\ = \{ \{A, B, C, D\} \}& \\
            \end{flalign*}
        \end{minipage}
    \end{figure}
    
    \begin{figure}[H]
    	\centering
        \begin{tikzpicture}[node distance = 2.2cm, ->, >=stealth, line width=1pt]
            \node[state, accepting, initial, initial text=] (a) {$I$};
           	\path (a) edge[loop] node[above]{$a, b$} (a);
           
        \end{tikzpicture}
        \caption{Redukovaný DKA}
    \end{figure}
    
\vspace{-15pt}
\subtask % (b)
Počet tried ekvivalencie relácie $\sim_{L}$ je len jedná jediná a to $L^{-1}([I]) = (a+b)^{*}$.
\begin{figure}[ht]
        \centering
        \begin{tikzpicture}[node distance = 2.5cm, ->, >=stealth, line width=1pt]
            \node[state, initial, initial text=, accepting] (f) {$[I]$};
			\path (a) edge[loop] node[above]{$a, b$} (a);
        \end{tikzpicture}
        \caption{KA príjmajúci jazyk $L^{-1}([I]) = (a+b)^{*}$.}
    \end{figure}
    
\end{mysolution}


\clearpage
\task{3}{Uvažujte NKA $M_{3}$ nad abecedou $\Sigma = \{a, b, c\}$ z obrázku \ref{KA3}: \\
    \begin{figure}[!h]
        \centering
        \begin{tikzpicture}[node distance = 3cm, ->, >=stealth, line width=1pt]
            \node[state, initial, initial text=] (X) {};
            \node[state, right of = X] (Y) {};
            \node[state, accepting, right of = Y] (Z) {};

            \path (X) edge node[above]{a} (Y);
            \path (Y) edge node[above]{b} (Z);
            \path (Y) edge[loop above] node[above]{c} (Y);
            \path (X) edge[bend right=40] node[below]{c} (Y);
            \path (Z) edge[bend left=60] node[above]{b} (X);
            \path (Z) edge[bend right=50] node[above]{a} (Y);
        \end{tikzpicture}
    \caption{NKA $M_{3}$}
    \label{KA3}
    \end{figure}

    K tomuto automatu zostrojte sústavu rovníc nad regulárnymi výrazmi a jej riešením zostavte
ekvivalentný regulárny výraz.}

\begin{mysolution}

Označíme si jednotlive stavy KA z ľava doprava písmenami $X, Y \text{a } Z$ a zostrojíme sústavu rovníc.
\begin{flalign}
    X =&\ aY + cY & \\
    Y =&\ cY + bZ & \\
    Z =&\ bX + aY + \varepsilon &
\end{flalign}

Z výrazu $X$ vytkneme $Y$ a na výraz $Y$ použijeme vetu 3.14, ktorá hovorí že najmenším pevným bodom rovnice $X = pX + q$ je $X = p^{*}q$.

\begin{flalign}
    X =&\ (a + c)Y & \\
    Y =&\ c^{*}bZ & 
\end{flalign}
Výraz $Y$ dosadíme do výrazov $X$ a $Z$.

\begin{flalign}
    X =&\ (a + c)c^{*}bZ & \\
    Z =&\ bX + ac^{*}bZ + \varepsilon &  \\
      =&\ (ac^{*}b)^{*}(bX + \varepsilon) &
\end{flalign}
Následne výraz $Z$ dosadíme do $X$ a upravujeme.

\begin{flalign}
    X =&\ (a + c)c^{*}b(ac^{*}b)^{*}(bX + \varepsilon) & \\
    	=&\ (a + c)c^{*}b(ac^{*}b)^{*}bX + (a + c)c^{*}b(ac^{*}b)^{*} & \\
    	=&\ ((a + c)c^{*}b(ac^{*}b)^{*}b)^{*}(a + c)c^{*}b(ac^{*}b)^{*} &
\end{flalign}

Výraz $X$ je ekvivalentný regulárny výraz k automatu $M_{3}$

\end{mysolution}

\task{4}{
Majme jazyk $L_{1} \subseteq \{a, b, c\}^{*}$ a jazyk $L_{2} \subseteq \{0, 1\}^{*}$ a
funkciu $bin : \{a, b, c\}^{*} \to \{0, 1\}$ definovanú následovne:
\begin{itemize}
	\item $bin(\varepsilon) = \varepsilon, bin(a) = 00, bin(b) = 01$ a $bin(c) = 11$.
	\item Pre $|w| > 1 $ platí: $(w = xw' \land |x| = 1) \Rightarrow bin(w) = bin(x)bin(w')$
\end{itemize}
Navrhnite a formálne popíšte algoritmus, ktorý má na vstupe dva konečné automaty $M_{1} = (Q_{1}, \{a, b, c\}, \delta_{1}, q^{0}_{1}, F_{1})$ a $M_{2} = (Q_{2}, \{0, 1\}, \delta_{2}, q^{0}_{2}, F_{1})$ (môžu byť nedeterministické), a jeho výstupom bude konečný automat $M_{f}$ taký, že $L(M_{f}) = \{w | w \in L(M_{1}) \land bin(w) \in L(M_{2})\}$.
}
\begin{mysolution}

    \vspace{-35pt}
    \begin{flalign*}
        &M_{f} = (Q_{f}, \Sigma_{f}, \delta_{f}, q^{0}_{f}, F_{f}) & \\
        & \\
        &Q_{f} = Q_{1} \times Q_{2} & \\
        &\Sigma_{f} = \{a, b, c\} & \\
        &\delta_{f} :
                \forall q_{1}, q_{2} \in Q_{1} \forall q_{3}, q_{4}, q_{5} \in Q_{2} \forall x \in \Sigma_{f} : & \\
                    &\hspace{140pt} (q_{2}, q_{5}) \in \delta_{f}((q_{1}, q_{3}), x)  & \\
                    &\hspace{180pt} \Longleftrightarrow & \\
                    &\hspace{100pt} (q_{2} \in \delta_{1}(q_{1}, a) \land q_{4} \in \delta_{2}(q_{3}, 0) \land q_{5} \in \delta_{2}(q_{4}, 0)) &\\
                    &\hspace{200pt} \lor & \\
                     &\hspace{100pt} (q_{2} \in \delta_{1}(q_{1}, b) \land q_{4} \in \delta_{2}(q_{3}, 0) \land q_{5} \in \delta_{2}(q_{4}, 1)) &\\
                    &\hspace{200pt} \lor & \\
                     &\hspace{100pt} (q_{2} \in \delta_{1}(q_{1}, c) \land q_{4} \in \delta_{2}(q_{3}, 1) \land q_{5} \in \delta_{2}(q_{4}, 1)) &\\
        & \\
        &q^{0}_{f} = (q^{0}_{1}, q^{0}_{2}) & \\
        &F_{f} = F_{1} \times F_{2}
    \end{flalign*}

\end{mysolution}

\task{5}{
Majme jazyk $L = \{a^{i}b^{j}c^{k} | i > k \land j \geq 1\}$. Je jazyk $L$ regulárny? Dokážte alebo vyvráťte.
}
\begin{mysolution}
 PPredpokladajme, že jazyk $L$ je regulárny jazyk.
    Potom platí, že
    \begin{align*}
        \exists k > 0 : \forall w \in L :
        \vert w \vert \ge k \Rightarrow
        \exists x,y,z \in \Sigma^{*} :&\ w = xyz\ \land \\
                                      &\ y \ne \varepsilon\ \land \\
                                      &\ \vert xy \vert \le k\ \land \\
                                      &\ \forall i \ge 0 : xy^{i}z \in L\
    \end{align*}
    Zvolíme si reťazec $w$:
    \begin{align*}
    	w =&\ a^{k}bc^{k-1}, w \in L & \\
     	|w| =&\ 2k \ge k &
    \end{align*}
    
    Reťazec $xy^{i}z \in L$ \\
    \begin{align*}
    	\exists x,y,z \in \Sigma^{*} : |xy| \le k \\
   	 	x = a^{l_1}, y = a^{l_2}, \#_{a}(z) = l_{3} \\ 
   	 	l_{1}, l_{2} > 0, l_{3} \ge 0 \\
    	\#_{a}(w) = l_{1} + l_{2} + l_{3} = k \\
    \end{align*}
    Kedže $l_{3} \ge 0$ tak potom $l_{1} + l_{2} \le k$ \\
    Zvolíme si $i = 0$
    \begin{align*}
    	xy^{0}z =&\ xz = a^{l_1}a^{l_3}bc^{k-1} & \\
    	|a^{l_1}a^{l_3}| =&\ k - l_{2}, |c^{k-1}| = k - 1 &
    \end{align*}
    	Zo zadania by malo platiť že $\#_{a}(w) > \#_{c}(w)$
    \begin{align*}
    	\#_{a}(xy^{0}z) =&\ l_{1} + l_{3} & \\
    	\#_{c}(xy^{0}z) =&\ k - 1 & \\
    	l_{1} + l_{3} >&\ k - 1 & \\
    	k - l_{2} >&\ k - 1 \qquad l_{2} > 0 & \\
    	l_{2} <&\ 1 &
    \end{align*}
    Kedže nám vyšlo $l_{2} < 1 \land l_{2} > 0$ pričom $l_{2} \in N$, tak takéto $l_{2}$ neexistuje, takže
    $xy^{0}z \notin L$, došli sme k sporu. Tým pádom naš predpoklad neplatí a jazyk L nie je regulárny jazyk.
\end{mysolution}

\task{6}{
Dokážte, že pre každý regulárny jazyk existuje jednoznačná gramatika (definícia jednoznačnej gramatiky -- viz slidy 4, strana 11).
}
\begin{mysolution}

\begin{enumerate}
	\item Každý regulárny jazyk $L$ sa dá špecifikovať pomocou konečneho automatu $M$ tak že $L(M) = L$. Veta 3.6
	\item Konečný automat $M$ sa dá previesť na regulárnu gramatiku G, tak že $L(G) = L(M)$. Veta 3.7
	\item
		\begin{enumerate}
			\item Predpokladajme že gramatika $G$ je viacznačná. $G = (Q_{G}, \Sigma_{G}, P, q_{G}^{0})$
			\item \label{def} Podľa definície 4.5 je gramatika $G$ viacznačná ak generuje aspoň jednu viacznačnú vetu $w$. Veta $w$ je generovaná gramatikou $G$ ku ktorej existujú aspoň 2 rôzne derivačné stromy s koncovými uzlami tvoriacu vetu $w$.
			\item \label{gn} Z dôkazu pre vetu 3.7 vieme že KA M sa dá previesť na DKA N. $N = (Q_{N}, \Sigma_{N}, \delta_{N}, q_{N}^0, F_{N})$
			\item Z \ref{def} teda platí $\forall r \in \Sigma_{G} \exists q \in Q_{G} \exists a \in \Sigma_{G} : |q \rightarrow ar| > 1$, potom platí $\exists q \in Q_{N} \exists a \in \Sigma_{N}: |\delta(q,a)| > 1$, čiže k odvodzovaciemu pravidlu gramatiky existuje ekvivalentný prechod v konečnom automate.
			\item Kedže na základe \ref{gn} musí pre KA existovať viac ako jeden prechod pre určitý stav a symbol, tak dochádzame k sporu. Kedže na základe definície DKA, môže existovať maximálne jeden takýto prechod. $\forall q \in Q_{N} \forall a \in \Sigma_{N}: |\delta(q,a)| \leq 1$
			\item Gramatika tedy neni viacznačná. Definícia 4.5 hovorí že ak gramatika neni viacznačná tak je potom jednoznačná.
		\end{enumerate}
	\item Gramatika G je jednoznačná. Takže pre každý regulárny jazyk existuje jednoznačná gramatika.
\end{enumerate}

\end{mysolution}

\end{document}
\grid
\grid
\grid
