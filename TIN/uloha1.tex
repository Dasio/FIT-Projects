% Šáblona prevzatá od Mareka Milkoviča
% https://github.com/metthal/TIN-Ulohy

\documentclass[12pt]{article}

\usepackage[margin=1in]{geometry} 
\usepackage{amsmath,amsthm,amssymb,amsfonts}
\usepackage[slovak]{babel}
\usepackage[utf8]{inputenc}
\usepackage{enumitem}
\usepackage{ifthen}
\usepackage{tikz}
\usepackage{tikz-qtree}
\usepackage{caption}
\usepackage{background}
\usepackage{float}

\usetikzlibrary{arrows,automata,calc,positioning}

\newcommand{\task}[2]{\par \noindent \textbf{{#1}.} \hspace{3pt} #2 \vspace{10pt}}
\newcommand{\solution}{\vspace{10pt}}
\newcommand{\pipesep}{\hspace{3pt} \vert \hspace{3pt}}

\newenvironment{subtasklist}[0]{\begin{enumerate}[label=(\alph*)]}{\end{enumerate}}
\newenvironment{mysolution}[1]{
    \par \textbf{Riešenie} \newline
    \ifthenelse{\equal{#1}{subtasks}}{\begin{enumerate}[label=(\alph*)]}
            {\begin{enumerate}[label={}] \item}
}{\end{enumerate} \newpage}
\newcommand{\subtask}{\item}

\SetBgContents{
        \parbox{0.4\textwidth}{
            \raggedleft
                Dávid Mikuš\\
                \texttt{xmikus15}
        }
}
\SetBgScale{0.75}
\SetBgOpacity{1}
\SetBgAngle{0}
\SetBgColor{black}
\SetBgPosition{current page.north east}
\SetBgVshift{-0.8cm}
\SetBgHshift{-3.5cm}
 
\begin{document}
 
\title{TIN 2017/2018: Úloha 1}
\author{Dávid Mikuš \\ \small\texttt{xmikus15@stud.fit.vutbr.cz}}
\maketitle

\task{1}{Nech $M_1 = (Q_1,\Sigma_1, \delta_1, \Gamma_1,q_1, Z_1, F_1)$ je zásobnikový automat a
$M_2 = (Q_2,\Sigma_2, \delta_2 ,q_2, F_2)$ je nedeterministický konečný automat.

Navrhnite a \textit{formálne popíšte} algoritmus ktorý má na vstupe automaty $M_1$ a $M_2$, a jeho
výstupom bude zásobnikový automat $M_3$ taký že,
\begin{center}
$L(M_3) = \{w \pipesep w \in L(M_1) \land \exists w' \in L(M_2): |w| = |w'| \}$
}
\end{center}
\begin{mysolution}

  \vspace{-35pt}
    \begin{flalign*}
        &M_{3} = (Q, \Sigma, \delta, \Gamma, q_0, Z, F) & \\
        & \\
        &Q = Q_{1} \times Q_{2} & \\
        &\Sigma = \Sigma_1 & \\
        &\delta :
                \forall q_{1}^1, q_{2}^1 \in Q_{1} \ \forall q_{1}^2, q_2^2 \in Q_{2} \ \forall a \in \Sigma_{1} \ \forall z \in \Gamma_1 \ \forall \chi \in \Gamma_1^* : & \\
                    &\hspace{100pt} [\ ((q_2^1, q_2^2), \chi) \in \delta((q_1^1, q_1^2), a, z)  & \\
                    &\hspace{160pt} \Longleftrightarrow & \\
                    &\hspace{100pt}  \exists b \in \Sigma_2: ((q_2^1, \chi) \in \delta_{1}(q_1^1, a, z) \land q_2^2 \in \delta_{2}(q_1^2, b)\ ] &\\
                    &\hspace{180pt} \cup & \\
                     &\hspace{100pt} [\ ((q_2^1, q_2^2), \chi) \in \delta((q_1^1, q_1^2), \varepsilon, z)) &\\
                    &\hspace{160pt} \Longleftrightarrow & \\
                     &\hspace{100pt} ((q_2^1, \chi) \in \delta_{1}(q_1^1, \varepsilon, z) \land q_1^2 = q_2^2) \ ] &\\
        & \\
        &q_0 = (q_1, q_2) & \\
        &\Gamma = \Gamma_1 \\
        &Z = Z_1 \\
        &F = F_{1} \times F_{2}
    \end{flalign*}
    
\end{mysolution}

\task{2}{Uvažujme binárne relácie nad jazyky $\circ$ definované následovne: $L_1 \circ L_2 = \overline{L_1} \cap \overline{L_2}$. S využitím uzáverovych vlastnosti dokážte, alebo vyvráte, nasledujúce vzťahy:
\begin{enumerate}[label=(\alph*)]
\item{$L_1, L_2 \in \mathcal{L}_3 \Rightarrow L_1 \circ L_2 \in \mathcal{L}_3$}
\item{$L_1 \in \mathcal{L}_3, L_2 \in \mathcal{L}_2^D \Rightarrow L_1 \circ L_2 \in \mathcal{L}_2^D$}
\item{$L_1 \in \mathcal{L}_3, L_2 \in \mathcal{L}_2 \Rightarrow L_1 \circ L_2 \in \mathcal{L}_2$}
\end{enumerate}
$\mathcal{L}_2^D$ značí triedu deterministických bezkontextových jazykov.
}

\begin{mysolution}

\begin{enumerate}[label=(\alph*)]
\item  Upravíme na tvar: $L_1, L_2 \in \mathcal{L}_3 \Rightarrow \overline{L_1} \cap \overline{L_2} \in \mathcal{L}_3$
Podľa vety 3.7 vieme že trieda regulárnych jazykov tvorí množinovu Boolovu algebru. Z Boolovej algebry plynie uzavretosť voči doplnku a prieniku. \\
Kedže $\overline{L_1}, \overline{L_2} \in \mathcal{L}_3 $ tak potom aj $\overline{L_1} \cap \overline{L_2} \in \mathcal{L}_3 $   \\
Tento vzťah teda \textit{platí}.

\item Upravíme na tvar: $L_1 \in \mathcal{L}_3, L_2 \in \mathcal{L}_2^D \Rightarrow \overline{L_1} \cap \overline{L_2} \in \mathcal{L}_2^D$
Na zǎklade vety 3.7 trieda regulárnych jazykov tvorí množinovu algebru ktora je uzavretá voči doplnku, teda $ \overline{L_1} \in \mathcal{L}_3 $. Podľa definície 6.5 vieme ak $L = L(P)$ kde P je deterministicky zásobnikovy automat, tak $L$ sa nazýva deterministickým bezkontextovým jazykom. Kedže $L_2 \in \mathcal{L}_2^D$ tak pre tento jazyk existuje DZA ktorý ho generuje. \\
Podľa vety 6.12 sú jazyky DZA uzavreté voči doplnku a prieniku s regulárnymi jazykmi.
Čiže $\overline{L_2} \in \mathcal{L}_2^D$ a $ \overline{L_1} \in \mathcal{L}_3 $, tak aj ich prienik je uzavretý $\overline{L_1} \cap \overline{L_2} \in \mathcal{L}_3$. \\
Tento vzťah teda \textit{platí}.

\item Upravíme na tvar: $L_1 \in \mathcal{L}_3, L_2 \in \mathcal{L}_2 \Rightarrow \overline{L_1} \cap \overline{L_2} \in \mathcal{L}_2$. $L_2$ je teda z triedy bezkontextových jazykov a na základe vety 6.6 vieme že bezkontextový jazyk nie je uzavretý voči doplnku a prieniku. $\overline{L_2} \notin \mathcal{L}_2$. Regulárne jazyky sú uzavreté voči doplnku, teda $\overline{L_1} \in \mathcal{L}_3$. Zvolíme si $\overline{L_1} = \{a,b,c\}^* \in \mathcal{L}_3$  a $\overline{L_2} = \{a^nb^nc^n | n \geq 1 \} \notin \mathcal{L}_2$. $ \overline{L_1} \cap \overline{L_2} = \{a^nb^nc^n | n \geq 1 \} \notin \mathcal{L}_2 $ \\
Našli sme protipríklad a teda tento vzťah \textit{neplatí}.

\end{enumerate}
\end{mysolution}


\task{3}{Nech $\Sigma = \{a, b, c\}}$. Uvažujme jazyk $L$ nad abecedou $\Sigma \cup \{1, 2\}$ definovaný následovne: $L = \{w_11w_2 \pipesep w_1, w_2 \in \Sigma^*, \#_{a}(w_1) = \#_{c}(w_2)\} \cup
\{w_12w_2 \pipesep w_1, w_2 \in \Sigma^*, \#_{b}(w_1) = \#_{c}(w_2)\}$
Zostrojte deterministický zásobnikový automat $M_L$ taký, že $L(M_L) = L$.   
\begin{mysolution}

$M_L = (\{q_0,q_1,q_2,q_3\}, \{a, b, c, 1, 2\}, \{\#, a, b\}, \delta, q_0, \#, \{q_3\})$
\begin{align*}
\delta(q_0, a, \#) &= \{(q_0, a\#)\} &
\delta(q_0, a, a) &= \{(q_0, aa)\} &
\delta(q_0, a, b) &= \{(q_0, ab)\} \\
\delta(q_0, b, \#) &= \{(q_0, b\#)\} &
\delta(q_0, b, a) &= \{(q_0, ba)\} &
\delta(q_0, b, b) &= \{(q_0, bb)\} \\
\delta(q_0, c, \#) &= \{(q_0, \#)\} &
\delta(q_0, c, a) &= \{(q_0, a)\} &
\delta(q_0, c, b) &= \{(q_0, b)\} \\
%
\delta(q_0, 1, \#) &= \{(q_1, \#)\} &
\delta(q_0, 1, a) &= \{(q_1, a)\} &
\delta(q_0, 1, b) &= \{(q_1, b)\} \\
\delta(q_0, 2, \#) &= \{(q_2, \#)\} &
\delta(q_0, 2, a) &= \{(q_2, a)\} &
\delta(q_0, 2, b) &= \{(q_2, b)\}
\end{align*}
\begin{align*}
\delta(q_1, \varepsilon, b) &= \{(q_1, \varepsilon)\} &
\delta(q_2, \varepsilon, a) &= \{(q_2, \varepsilon)\} \\
\delta(q_1, c, a) &= \{(q_1, \varepsilon)\} &
\delta(q_2, c, b) &= \{(q_2, \varepsilon)\} \\
\delta(q_1, a, a) &= \{(q_1, a)\} &
\delta(q_2, a, b) &= \{(q_2, b)\} \\
\delta(q_1, b, a) &= \{(q_1, a)\} &
\delta(q_2, b, b) &= \{(q_2, b)\} \\
\delta(q_1, \varepsilon, \#) &= \{(q_3, \#)\} &
\delta(q_2, \varepsilon, \#) &= \{(q_3, \#)\} \\
%
\delta(q_3, a, \#) &= \{(q_3, \#)\} &
\delta(q_3, b, \#) &= \{(q_3, \#)\}
\end{align*}

 \begin{figure}[!h]
        \centering
        \begin{tikzpicture}[node distance = 3cm, ->, >=stealth, line width=1pt, auto]
            \node[state, initial, initial text=\#] (qa) {$q_0$};
            \node[state, above right of = qa] (qb) {$q_1$};
            \node[state, below right of = qa] (qc) {$q_2$};
            \node[state, accepting, below right of = qb] (qd) {$q_3$};
            \path (qa) edge[in=200, out=270, looseness=10] node[left=0.3]{
            $\begin{aligned}
            a, \#&/a\# &
           	a, a&/aa &
           	a, b&/ab \\
           	b, \#&/b\# &
           	b, a&/ba &
           	b, b&/bb \\
           	c, \#&/\# &
           	c, a&/a &
           	c, b&/b
            \end{aligned}$
            } (qa);
            \path (qa) edge[bend left=40] node[above=0.3]{
            $\begin{aligned}
            1, \#/\# \\
            1,a/a \\
            1,b/b
            \end{aligned}$
            } (qb);
             \path (qa) edge[bend right=40] node[below=0.5]{
            $\begin{aligned}
            2, \#/\# \\
            2,a/a \\
            2,b/b
            \end{aligned}$
            } (qc);
             \path (qb) edge[in=45, out=90, looseness=10] node[right=0.3]{
            $\begin{aligned}
            \varepsilon, b&/\varepsilon &
            c,a&/\varepsilon \\
            a,a&/a &
            b,a&/a
            \end{aligned}$
            } (qb);
             \path (qc) edge[in=0, out=270, looseness=5] node[right=0.3]{
            $\begin{aligned}
            \varepsilon, a&/\varepsilon &
            c,b&/\varepsilon \\
            b,b&/b &
            a,b&/b
            \end{aligned}$
            } (qc);
            \path (qb) edge node[above, sloped]{$\varepsilon, \#/\varepsilon$} (qd);
            \path (qc) edge node[above, sloped]{$\varepsilon, \#/\varepsilon$} (qd);
            \path (qd) edge[loop right] node[right]{
            $\begin{aligned}
          	a, \#&/\# \\
          	b, \#&/\#
            \end{aligned}$
            } (qa);
        \end{tikzpicture}
        \caption{Deterministický zásobnikový automat $M_L$}
    \end{figure}
\end{mysolution}
\task{4}{Dokážte, že jazyk L z predchádzajúceho príkladu není regulárny.
}
\begin{mysolution}
 PPredpokladajme, že jazyk $L$ je regulárny jazyk.
    Potom platí, že
    \begin{align*}
        \exists k > 0 : \forall w \in L :
        \vert w \vert \ge k \Rightarrow
        \exists x,y,z \in \Sigma^{*} :&\ w = xyz\ \land \\
                                      &\ y \ne \varepsilon\ \land \\
                                      &\ \vert xy \vert \le k\ \land \\
                                      &\ \forall i \ge 0 : xy^{i}z \in L\
    \end{align*}
    Zvolíme si reťazec $w$:
    \begin{align*}
    	w =&\ a^{k}1c^{k}, w \in L & \\
     	|w| =&\ 2k+1 \ge k &
    \end{align*}
    Kedžte platí 
    $\ \vert xy \vert \le k$, tak reťazec $xy$ sa skladá výhradne so symbolov $a$. Môžeme zapísať
    $xy^iz  = a^{k + (i - 1)|y|}1c^k$. Pre $i = 0$ teda platí $xy^0z = a^{k - |y|}1c^k$. Z toho vyplýva že
    $\#_a( a^{k - |y|}1c^k) = \#_c( a^{k - |y|}1c^k)$. Zostavíme rovnicu $k - |y| = k$, čiže $-|y| = 0$ a to platí len pre $|y| = 0$ a kedže vieme že $ y \ne \varepsilon, |y| > 0$. Tak dochádzame k sporu. \\
    Pôvodný predpoklad, že jazyk L je regulárny teda nie je
platný. Jazyk L nie je regulárny
\end{mysolution}

\task{5}{Uvažujme jazyk $L_k$ definovaný následovne: $L_k = \{w_1 \# w_2 \pipesep w_1, w_2 \in \{0, 1, 2\}^*, \#_0(w_1) < \#_2(w_2) < \#_1(w_1)\}$. Dokažte, že $L_k$ nie je bezkontextový.
}
\begin{mysolution}
NNech je	$L_k$ bezkontextový jazyk. Potom platí
	\begin{align*}
		\exists k > 0 : \forall z \in L_k :
		\vert z \vert \ge k \Rightarrow
		\exists u,v,w,x,y \in \Sigma^{*} :&\ z = uvwxy\ \land \\
									  &\ vx \ne \varepsilon\ \land \\
									  &\ \vert vwx \vert \le k\ \land \\
									  &\ \forall i \ge 0 : uv^{i}wx^{i}y \in L_k\
	\end{align*}
	Nech reťazec $z = 0^{k-1}1^{k+1}\#2^{k}$. Určite platí $|z| \ge k$, lebo vieme, že
	$|z| = 3k$. Nakoľko $|vwx| \le k$, tak môže nastať len istý počet možností toho, v akom
	tvare je $vwx$. Tie sú:
	\begin{enumerate}
	\item $vwx$ sa skladá výhradne zo symbolov $0$
	\item $vwx$ sa skladá na rozhraní symbolov $0,1$
	\item $vwx$ sa skladá výhradne zo symbolov $1$
	\item $vwx$ sa skladá zo symbolov $1,2$ a $\#$
	\end{enumerate}
	Prvý prípad môžeme prepísať na ekvivalentný tvar $0^{k-1+(i-1)|vx|}1^{k+1}\#2^{k}$. \\
	Položme $i = 2$, teda musí platiť $k -1 +|vx| < k$. Kedže $ vx \ne \varepsilon, |vx| > 0 $ tak sme v tejto variante došli k sporu. \\
	
	V druhom prípade sa vo $v$ alebo $x$ bude nachádzať aspoň jeden symbol $1$, pre $i = 0$ teda aspon jeden symbol $1$ vypadne a počet symbolov sa zmení z $k + 1$ na $k$ a tu dochádzame k sporu pretože $k \nless k$. \\
	
	Tretí prípad podobne ako prvý môžeme prepísať na ekvivalentný tvar $0^{k-1}1^{k+1+(i-1)|vx|}\#2^{k}$. 
	Položme $i = 0$, teda musí platiť $k < k + 1 - |vx|$. Kedže $ vx \ne \varepsilon, |vx| > 0 $ tak sme v tejto variante došli k sporu. \\	
	
	Vo štvrtom prípade, ak by sa vo $v$ alebo $x$ nachádzal symbol $\#$ tak dojdeme k sporu pre $i \neq 1$, kedže $\#$ sa nachádza v jazyku práve jeden krát. \\
	Uvažujme všetky možnosti $vwx$ také že: 
	$a = \#_1(vwx), b = \#_2(vwx), a \geq 0 \ b > 0$
	\begin{enumerate}
	\item $a > b$ - podobne ako v prípade 3 pre $i = 0$ dojdeme k sporu, pretože dôjde k narušeniu nerovnosti počtu $1$ a $2$
	\item $a < b$ - pre $i > 1$ dôjdeme k sporu, pretože sa zvýši počet $2$
	\item $a = b$ - pre $i = 0$ dôjdeme k sporu, pretože sa zníži počet $2$
	\end{enumerate}
	Nakoľko sme sa dostali v každom
	prípade do sporu, tak aj pôvodné tvrdenie, že $L_k$ je bezkontextový jazyk nemôže platiť. Jazyk $L_k$ tým
	pádom nie je bezkontextový.

	
\end{mysolution}

\task{6}{
Popíšte hlavnú ideu dôkazu, že pre každý regulárny jazyk existuje jednoznačná gramatika (definícia jednoznačnej gramatiky -- viz slidy 2, strana 11).
}
\begin{mysolution}

\begin{enumerate}
	\item Každý regulárny jazyk $L$ sa dá špecifikovať pomocou konečneho automatu $M$ tak že $L(M) = L$. Veta 3.6
	\item Konečný automat $M$ sa dá previesť na regulárnu gramatiku G, tak že $L(G) = L(M)$. Veta 3.7
	\item
		\begin{enumerate}
			\item Predpokladajme že gramatika $G$ je viacznačná. $G = (Q_{G}, \Sigma_{G}, P, q_{G}^{0})$
			\item \label{def} Podľa definície 4.5 je gramatika $G$ viacznačná ak generuje aspoň jednu viacznačnú vetu $w$. Veta $w$ je generovaná gramatikou $G$ ku ktorej existujú aspoň 2 rôzne derivačné stromy s koncovými uzlami tvoriacu vetu $w$.
			\item \label{gn} Z dôkazu pre vetu 3.7 vieme že KA M sa dá previesť na DKA N. $N = (Q_{N}, \Sigma_{N}, \delta_{N}, q_{N}^0, F_{N})$
			\item Z \ref{def} teda platí $\forall r \in \Sigma_{G} \exists q \in Q_{G} \exists a \in \Sigma_{G} : |q \rightarrow ar| > 1$, potom platí $\exists q \in Q_{N} \exists a \in \Sigma_{N}: |\delta(q,a)| > 1$, čiže k odvodzovaciemu pravidlu gramatiky existuje ekvivalentný prechod v konečnom automate.
			\item Kedže na základe \ref{gn} musí pre KA existovať viac ako jeden prechod pre určitý stav a symbol, tak dochádzame k sporu. Kedže na základe definície DKA, môže existovať maximálne jeden takýto prechod. $\forall q \in Q_{N} \forall a \in \Sigma_{N}: |\delta(q,a)| \leq 1$
			\item Gramatika tedy neni viacznačná. Definícia 4.5 hovorí že ak gramatika neni viacznačná tak je potom jednoznačná.
		\end{enumerate}
	\item Gramatika G je jednoznačná. Takže pre každý regulárny jazyk existuje jednoznačná gramatika.
\end{enumerate}

\end{mysolution}

\end{document}

\grid
